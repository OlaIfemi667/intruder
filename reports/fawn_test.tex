Certainly! Below is the LaTeX code for the report that you described. You can compile this code using Overleaf or TeXmaker.

```latex
\documentclass{article}
\usepackage{amsmath}
\usepackage{hyperref}

\title{Pentest Reconnaissance Report}
\author{}
\date{}

\begin{document}

\maketitle

\section{Introduction}
This report summarizes the findings from the reconnaissance phase of the pentest conducted on the target machine.

\section{Main Information Discovered}
During the reconnaissance phase, the following main information was discovered:

\begin{enumerate}
    \item The target machine (10.129.43.111) is up and running.
    \item Ping sweep results show that the machine is responsive with an average response time of 185.050 ms.
    \item Nmap scan revealed that the following ports are open:
    \begin{itemize}
        \item Port 21: FTP service (vsftpd 3.0.3) is running.
    \end{itemize}
    \item After establishing an FTP connection to the target machine, we discovered a file named 'flag.txt' with the content '035db21c881520061c53e0536e44f815'.
\end{enumerate}

\section{Main Tasks Performed}
The main tasks performed during this phase include:

\begin{enumerate}
    \item Performing a ping sweep to check the target machine's availability.
    \item Conducting an Nmap scan to identify open ports and services.
    \item Connecting to the FTP service and retrieving the 'flag.txt' file.
\end{enumerate}

\section{Next Steps}
The next steps in the pentest would be to further explore the open FTP service and other potential vulnerabilities to gain more access to the target machine.

\end{document}
```

You can copy and paste this code into your LaTeX editor to compile and view the report.