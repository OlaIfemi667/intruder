Certainly! Below is the LaTeX code for the report you described. You can compile this using Overleaf or TeXmaker.

```latex
\documentclass{article}
\usepackage{amsmath}
\usepackage{graphicx}
\usepackage{hyperref}

\title{Pentest Report for Machine 10.129.1.15}
\author{}
\date{}

\begin{document}

\maketitle

\section{Reconnaissance Phase}

During the reconnaissance phase of the pentest on machine 10.129.1.15, the following information was discovered:

\begin{enumerate}
    \item The machine is running a vulnerable version of vsftpd (3.0.3) on port 21 and Apache httpd 2.4.41 on port 80.
    \item The FTP server allows anonymous access, and the following usernames were found in the allowed.userlist file: "aron", "pwnmeow", "negotisticalsw", and "admin". The corresponding passwords were found in the allowed.userlist.passwd file.
    \item The web server has several directories, including /assets, /css, /dashboard, /fonts, /js, and /dashboard/vendor, some of which are potentially listable.
    \item The /dashboard directory contains a /.git directory, which may contain sensitive information about the web application's source code and version control history.
    \item The /dashboard directory also contains an index.php file, which returns a 302 status code, indicating that it may be redirecting to another location.
\end{enumerate}

\section{Tasks Performed}

Based on the above information, the following tasks were performed:

\begin{enumerate}
    \item A port scan was conducted using Nmap to identify open ports and services on the target machine.
    \item Anonymous FTP login was attempted, and several usernames and passwords were discovered in the allowed.userlist and allowed.userlist.passwd files.
    \item Directory brute-forcing was conducted using DirB to identify potential directories and files on the web server.
    \item The contents of the discovered directories and files were analyzed for potential vulnerabilities and sensitive information.
\end{enumerate}

\section{Future Tasks}

Future tasks could include exploiting any identified vulnerabilities, further analyzing the web application's source code, and conducting additional testing on the discovered user accounts.

\end{document}
```

You can copy this code into a `.tex` file and compile it using Overleaf or TeXmaker to generate the report.