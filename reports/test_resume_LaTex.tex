Certainly! Below is the LaTeX code for the report:

```latex
\documentclass{article}
\usepackage{amsmath}
\usepackage{geometry}
\geometry{a4paper, margin=1in}
\usepackage{hyperref}
\usepackage{graphicx}

\title{Reconnaissance Phase Report}
\author{}
\date{}

\begin{document}

\maketitle

\section{Introduction}
This report summarizes the main information discovered and tasks performed during the reconnaissance phase based on the terminal input and output provided.

\section{Ping Sweep}
A ping sweep was performed on localhost using the command \texttt{ping -c 4 localhost}. The output shows that all 4 ICMP packets were transmitted and received, with no packet loss, indicating that the localhost is up and running.

\begin{itemize}
    \item \textbf{Timestamp:} 2024-08-08 14:14 WAT
    \item \textbf{Target:} localhost
    \item \textbf{Result:} Localhost is up, with no packet loss.
\end{itemize}

\end{document}
```

You can copy and paste this code into Overleaf or TeXmaker to compile the report. This LaTeX document includes a title, author, date, and sections for the introduction and ping sweep details. The `\texttt` command is used to format the terminal command in a monospaced font, and the `itemize` environment is used to list the details of the ping sweep.